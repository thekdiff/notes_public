% Created 2016-09-22 周四 12:06
\documentclass[11pt]{article}
\usepackage[utf8]{inputenc}
\usepackage[T1]{fontenc}
\usepackage{fixltx2e}
\usepackage{graphicx}
\usepackage{longtable}
\usepackage{float}
\usepackage{wrapfig}
\usepackage{rotating}
\usepackage[normalem]{ulem}
\usepackage{amsmath}
\usepackage{textcomp}
\usepackage{marvosym}
\usepackage{wasysym}
\usepackage{amssymb}
\usepackage{hyperref}
\tolerance=1000
\author{xuzhixin}
\date{\today}
\title{cpp}
\hypersetup{
  pdfkeywords={},
  pdfsubject={},
  pdfcreator={Emacs 24.3.1 (Org mode 8.2.10)}}
\begin{document}

\maketitle
\tableofcontents

\begin{itemize}
\item strlen(char*)就算字符串的长度,返回直到'$\backslash$0'的长度(不包括'$\backslash$0'),如果找不
到'$\backslash$0',它会一直找下去
\item 严格的说,每个指针在初始化前都是野指针。delete ptr; ptr = NULL//删除
ptr指向的内存ptr如果不设为NULL就会成为野指针
\item 溢出问题
\item int arr[] = \{1, 2\}; int arr2[] = \{3, 4\}; arr2 = arr;//不合法
\item +优先级高于<<
\item c默认const是外部连接的,c++默认const是内部连接的
\item 静态变量存在全局数据区,sizeof只计算栈中分配的大小
\item strlen和sizeof的区别:strlen的参数必须是char*,且以'$\backslash$0'结尾。sizeof是运
算符,strlen是函数。p56,面试宝典
\item 数组传递给sizeof不会退化成指针
\item sizeof里的内容在编译时不被编译,只是被相应类型替换:a=8;siof(a=6);//a还
是等于8
\item 虚继承和虚函数都涉及一个虚函数表,就要为类生成一个指针,sizeof此类大小
会加4,p60,面试宝典
\item inline必须与函数定义放在一起才能成为内联函数,和函数声明放一起不行
\item extern C,实现C和C++的混合编程。extern表明变量声明,而非定义
\item 注意函数如果返回类型不为void,则结尾必须要返回一个值:仔细看
leetcode-211!!!!
\item string str = "xuzhixin";//以"xuzhixin"初始化str,不包括结尾的空字符
\item 类的静态成员函数才能用于sort中的比较函数:leetcode-179
\item sort(a.begin(), a.end(), compare);//compare定义了比较规则
\item malloc/free是库函数,只能分配内部数据类型;new/delete是运算符,可以自动
为类对象执行构造和析构函数
\item 深拷贝与浅拷贝
\item cin带逗号的输入,见test\_ cin.cpp: int a, b; cin >> a; cin.ignore(1,
','); cin >> b;//要包含头文件\#include<cstdlib>
\item string str; cin >> str;//遇到空格终止;getline(cin, str);//遇到空格不终
止,enter终止;cout << str; //输出整个str
\item 解引用*比加号+优先级低
\end{itemize}

\section{字符串,向量和数组}
\label{sec-1}
\subsection{vector}
\label{sec-1-1}
\begin{itemize}
\item vector 能容纳大多数类型的对象作为其元素,但是引用不是对象,故不存在包含
引用的vector
\item vector 对象的类型总是包含着元素的类型 vector<int>::size\_ type 正确
\item 两个vector对象比较时,所包含的元素必须是可比较的类生成的对象
\item 缓冲区溢出(buffer overflow):通过下标访问不存在的元素(数组,string,vector 等)
\item vector限制:1. 不能再范围for循环中向vector对象添加元素;2. 任何一种改变
vector对象容量的操作,如push\_ back都会使该vector对象的迭代器失效
\end{itemize}

\subsection{迭代器}
\label{sec-1-2}
\begin{itemize}
\item string对象不属于容器,但支持很多与容器类型相似的操作
\item 如果容器为空,则成员begin和end返回同一个迭代器,都是尾后迭代器
\item begin 和end返回的类型有对象是否是常量决定,如果是,返回const\_ iterator
否则返回iterator
\item 无论容器(string)对象本身是否是常量,返回值都是const\_ iterator
\item 但凡使用了迭代器的循环体,都不要向迭代器所属的容器中添加元素
\item 所有标准库容器的迭代器都只吃递增运算和比较运算符!= 和 ==
\end{itemize}

\subsection{数组}
\label{sec-1-3}
\begin{itemize}
\item 声明时的维度必须是 \textbf{常量表达式}
\item 数组的元素应为对象,故不存在引用的数组
\item 初始值可少不可以多
\item 字符数组的特殊性:char a[] = "C++",a的维数为4
\item 不能将数组的内容拷贝给其它数组作为初始值,也不能用数组为其他数组赋值
\item int *ptr\footnote{DEFINITION NOT FOUND.} 是含有十个整型指针的数组
\item int \&refs\footnotemark[1]{} 错误,不存在引用的数组
\item int (*Parray)\footnotemark[1]{} = \&arr ,指向一个含有十个整数的数组
\item int (\&arrRef)\footnotemark[1]{} = arr,引用一个含有十个整数的数组
\item int a[] = \{0,1\};auto a2(a); //a2是一个整型指针,指向a的第一个元素
\item decltype(a) a3 = \{0,1\}; //a3是一个两个整型(和a维数相同)数组
\item 容器的迭代器支持的运算,数组的指针全部支持
\item 尾后指针不能执行解引用和递增操作
\item p107,c++primer
\item 不允许数组和vector对象为内置类型的数组初始化,但是可以使用数组来初始化
vector对象
\item c++中没有多维数组,多维数组其实就是数组的数组,第一维行,第二维列
\item 为了防止数组被自动转成指针,使用范围for语句处理多维数组时,除了最内层的
循环外,其它所有循环的控制变量都应该是引用类型 p114,c++primer
\item 当程序使用多维数组的名字时,也会自动将其转换成指向数组首地址的指针
\item 别名:using intArray = int\footnote{DEFINITION NOT FOUND.} ; typedef int intArray\footnotemark[2]{};等价
\end{itemize}

\section{表达式}
\label{sec-2}
\subsection{基础}
\label{sec-2-1}
\begin{itemize}
\item C++定义了一元二元三元运算符,函数调用也是一种特殊的运算符,他对运算对象
的数量没有限制
\item 右值使用表达式的值,左值使用表达式的身份
\item 使用重载运算符时,包括运算对象的类型和返回值的类型,都是由该运算符定义
的;但是运算对象的个数、运算符的优先级和结合律都是无法改变的
\item 复合表达式是指含有两个或多个运算符的表达式
\item 对于没有指定执行顺序的运算符,如果表达式指向并修改了同一个对象,将会引
发错误并产生未定义的行为:cout<<i<<" "<<++i<<endl;//未定义的
\item 两条准则:1. 使用括号;2. 如果改变了某个运算对象的值,在表达式的其它地
方不要再使用这个运算对象(当改变运算对象的子表达式本省就是另外一个子表
达式的运算对象时,该规则无效,如*++iter)
\item 算术运算符的运算对象和求值结果都是右值,在表达式求值之前,小整数类型的
运算对象被提升成较大的整数类型,所有运算对象最终会转换成同一类型
\item 对大多数运算符来说,布尔类型的运算对象被提升成int类型
\item 整数相除(/)结果还是整数,参与取余运算的运算对象必须是整数类型
\item 两个运算对象的符号相同则为正(如果不为0的话),否则为负;C++11新标准规
定商一律向零取整(即直接切除小数部分)
\item m,n为整数,则(m/n)*n+m\%n==m;(-m)/n,m/(-n)都等于-(m/n),
m\%(-n)==m\%n,(-m)\%n==-(m\%n)
\item 关系运算符作用于算术或指针类型,逻辑运算符作用于任意能转换成布尔值得类
型,返回值都是布尔类型,运算对象和求值结果都是右值;值为0的运算对象(算
术或指针类型)表示假,否则为真
\item i<j<k 应该写成(i<j\&\&j<k)
\item 对于赋值运算符,如果左侧对象是内置类型,那么初始值列表最多只能包含一个
值,而且该值类型所占空间不应该大于目标类型的空间:int k;k=\{3.14\}//错误;
初始值列表总是可以为空
\item 对于多重赋值语句的每一个对象,它的类型或者与右边对象的的类型相同,或者
可有右边对象的类型转换得到:int ival, *pval;ival=pval=0//错误!!!!
\item 尽量使用前置版本的递增递减运算符(--i,++i),后置版本会增加开销
\item 后置递增运算符优先级高于解引用运算符(*iter++)
\item 箭头运算符作用于一个指针类型的运算对象,结果是一个左值。点运算符,如果
成员所属的对象是个左值,那么结果是左值,否则结果是右值
\item 对于条件运算符,当两个表达式都是左值或者能转换成同一种左值类型时,运算
结果是左值,否则是右值
\item 条件运算符优先级非常低
\item 位运算符作用于整数类型的运算对象,强烈建议只将位运算符用于处理无符号类
型;char类型的运算对象首先被提升成int类型
\item 1UL<<27,将无符号的1左移27位,结果只有其第27位上是1,其余为0
\end{itemize}
\subsection{sizeof运算符}
\label{sec-2-2}
\begin{itemize}
\item 返回一个表达式或一个类型名字所占的字节数,满足右结合律,值为size\_ t类型
的常量表达式,形式:sizeof (type);sizeof expr//返回表达式结果类型的大小。
\item sizeof并不实际计算其运算对象的值
\item Sales\_ data data, *p;sizeof(Sales\_ data)==sizeof(data)==sizeof(*p)//返
回Sales\_ data类型对象所占空间的大小;sizeof(p);//指针所占空间的大小,
32位系统为4,4位系统为8;sizeof data.revenue==sizeof Sales\_
data::revenue//revenue成员对应类型的大小
\item 因为sizeof不会实际求运算对象的值,故可以作用于无效指针(未初始化);在
sizeof的运算对象中解引用一个无效指针也是安全行为,因为指针实际上未被真
正使用,sizeof不需要真正解引用指针也能知道他所指对象的类型
\item sizeof:char-1,int-4,float-4,double-8,long-4,long long-8
\item 对引用类型执行sizeof得到所指向对象所占空间的大小
\item 数组:等价于对数组中所有元素各执行一次sizeof并求和,sizeof运算不会把数
组转化成指针来处理
\item 对string或vector对象求执行sizeof,只返回该类型固定部分的大小,不会计算
对象中的元素占用了多少空间;string-4,vector<int>-12,vector<char>-12
\end{itemize}
\subsection{类型转换(在仔细看)}
\label{sec-2-3}
\begin{itemize}
\item 表达式中既有整型又有浮点型运算对象,整型会转换成浮点型
\item 强制类型转换:cast-name<type>(expr)。cast-name:static\_ cast,dynamic\_
cast,const\_ cast,reinterpret\_ cast。
\item static\_ cast:只要不包含底层const都可以使用,当把较大的算术类型赋给较小
的时比较有用,以及编译器无法自动执行的类型转换。void * p=\&d;double
dp=static\_ cast<double*>(p);//将void*转换回初始的指针类型,任何非常量对
象的地址都能存入void*
\item const\_ cast:只能改变运算对象的底层const(且只有它能改变),常用于函数
重载的上下文中,必须运算对象本身不是常量,否则产生未定义的后果
\end{itemize}

\section{语句}
\label{sec-3}
\begin{itemize}
\item 空语句:;//空语句:while (cin>>s\&\&s!=sought) ; //空语句和while循环一体
\item 符合语句也称为快,一个快就是一个作用域,快不以分号作为结束
\item switch语句首先对括号里的表达式求值,如果表达式和某个case标签的值匹配成
功,程序从该标签后的第一条语句开始执行, \textbf{直到到达了switch的结尾} 或者是遇
到了一条break语句为止
\item case标签必须是整型常量表达式,任意两个case标签的值不能相同
\item p163,c++primer !!!!!!!!!!!!!!!!!!!
\item 定义在while条件部分或者while循环体内的变量每次迭代都经历从创建到销毁的
过程
\item for(init-statement;condition;expression) statement //和其他声明一样,
init-statement也可以定义多个对象,但是init-statement只能有一条声明语句,
故所有变量的基础类型必须相同
\item 范围for语句中,预存了end()的值,一旦在序列中增加(删除)元素,end函数的
值就可能变得无效了
\item do statement while(condition)//condition不能为空,condition使用的变量必
须定义在循环体之外
\item 只有当switch语句嵌套在迭代语句内部时,才能在switch里使用continue
\item goto label;label:\ldots{}//不要在程序中使用goto
\item 看异常处理
\end{itemize}

\section{函数}
\label{sec-4}
\begin{itemize}
\item 函数是一个命了名的代码块,通过调用函数执行相应的代码。可以重载函数,即
同一个名字可以对应几个不同的函数
\item 函数:1. 返回类型;2. 函数名字;3. 0个或多个参数组成的列表;4. 函数体
\item 编译器能以任意可行的顺序对实参求值
\item 实参和形参类型相同,或实参类型能转换成形参类型
\item 使用void显示定义空形参列表:void f(void)\{\}
\item 形参列表中的形参通常用逗号隔开,其中每个形参都是含有一个声明符的声明:
int f(int v1, v2)//错误
\item 任意两个形参不能同名,函数最外层的局部变量也不能使用与函数形参一样的名
字
\item 返回类型可以是void,表示不返回任何值。函数的返回值不能是数组类型或函数
类型,但可以是指向数组或函数的指针(数组的引用)
\item 名字有作用域,对象有生命周期。作用域:程序文本的一部分,名字在其中可见;
生命周期:是程序执行过程中该对象存在的一段时间
\item 我们把只存在于快执行期间的对象称为自动对象,形参是一种自动对象
\item 局部静态变量static,在程序执行路径第一次经过对象定义语句时初始化,并且
直到程序终止才被销毁(之后经过对象定义语句时忽略?)
\item 形参和函数体内部定义的变量统称为局部变量
\item 函数只能定义一次,但可以声明多次,函数声明无需函数体,只需;代替
\item 函数三要素:1. 返回类型,2. 函数名,3. 形参类型。描述了函数的接口,函数
声明也称函数原型
\item \textbf{分离式编译} p186,C++primer
\end{itemize}

\subsection{参数传递}
\label{sec-4-1}
\begin{itemize}
\item 形参初始化的机理与变量初始化一样
\item 参数传递分:1. 传值参数(包括指针形参), 2. 传引用参数
\item C++中,建议使用引用类型的形参替代指针
\item 使用引用避免拷贝,如果函数无需改变引用形参的值,最好将其声明为常量引用
\item 使用引用形参返回额外信息
\item 当形参有顶层const时,传给他常量对象和非常量对象都是可以的(即忽略掉顶层
const)
\item 可以定义具有相同名字的函数,但是不同函数的形参列表必须有明显的区别:
void fcn(const int i)\{\} void fcn(int i)\{\}//错误,重复定义了fcn(int)
\item int \& r = 42;//错误,不能用自面值初始化一个常量引用
\item C++允许我们用自面值初始化常量引用
\item 把函数不会改变的形参尽量定义成常量引用。非常量引用限制了函数所能接受的
实参类型。我们不能把const对象,字面值或者 \textbf{需要类型转换的对象} 传递给普
通的引用形参
\item 数组不允许拷贝和赋值;使用数组时(通常)会将其转换成指针
\item 数组:int j\footnote{DEFINITION NOT FOUND.} = \{0, 1\}; fcn(begin(j), end(j));//针对数组的begin和end函
数
\item 如果我们传给fcn函数的是一个数组,则实参自动的转换成指向数组首元素的指针,
数组的大小对函数的调用没有影响:void print(const int*);void
print(const int[]);void print(const int\footnotemark[1]{});三者等价。
\item 以数组作为形参的函数在使用数组时确保不要越界
\item 因为数组以指针的方式传给函数,所以一开始并不知道数组的尺寸,调用者应该
为此提供一些额外的信息:1. 使用标记指定数组的长度('$\backslash$0');2. 使用标准
库规范(begin(),end());3. 显示传递一个表示数组大小的形参
\item 把数组引用作为形参:void print(int (\&arr)\footnotemark[1]{})\{for (auto elem:arr)
cout<<elem<<endl;\}//\&arr两端的括号必不可少;需要指明数组的大小
\item main:命令行处理选项 p196,C++primer
\item 含有可变形参的函数:1. initializer\_ list形参;2. 省略符形参:仅仅用于C
和C++通用的类型
\item return语句的两种形式:1. return; 2. return expr; \textbf{注意} 不存在 return
void 形式
\item 无返回值函数:返回类型是void的函数,返回void的函数不要求非得要return语
句,这类函数的最后一句后面会隐式的执行return
\item 返回类型时void的函数使用return语句的第二种形式:expr必须是另一个返回
void的函数。强行令void函数返回其它类型的表达式会产生错误
\item 如果函数的返回类型不是void,则该函数内的每条return语句必须返回一个值
(编译器会检查)。return语句的返回类型与函数的返回类型相同,或者能隐式
的转换成函数的返回类型
\item 返回一个值的方式与初始化一个变量或形参的方式完全一样
\item 不要返回局部对象的引用或指针
\end{itemize}

\subsection{函数重载}
\label{sec-4-2}
\begin{itemize}
\item 同一作用域内几个函数名字相同但形参列表不同,我们称之为重载函数,main函
数不能重载
\item 对于重载函数,他们应该在形参数量或形参类型上有所不同, \textbf{不允许两个函数
除了返回类型外其它所有的要素都相同} (第二个函数声明时错误的)
\item 顶层const不影响传入函数的对象。一个拥有顶层const的形参无法与一个没有顶
层const的形参区分开来:Record lookup(Phone);Record lookup(const
Phone);重复声明了;Record lookup(Phone *);Record lookup(Phone
*const);//重复声明
\item 如果形参是某种类型的指针或引用,则通过区分其指向的是常量对象还是非常量
对象可以实现函数重载,此时的const是底层的
\item 是否重载函数要看是否更容易理解
\item const\_ cast在重载函数的场景中最有用const\_ cast<const
string\&>(s);const\_ cast<string\&>(s);
\item 调用重载函数时有三种结果:1. 编译器找到一个与实参 \textbf{最佳匹配} 的函数,并
生成调用该函数的代码;2. 找不到,此时编译器发出 \textbf{无匹配} 的错误信息;
\begin{enumerate}
\item 找到多余一个的函数可以匹配,但是每个都不是最佳选择, \textbf{二义性调用}
\end{enumerate}
\item 对于新作用域中的函数,隐藏外层定义域中的函数(或变量)
\item 在C++中,名字查找发生在类型检查之前
\item p211,C++primer
\end{itemize}
\subsection{特殊用途语言特性}
\label{sec-4-3}
\begin{itemize}
\item 可以为一个或多个形参定义默认值,一旦某个形参被赋予了默认值,其后的形参
必须都有默认值(否则编译错误)
\item 函数调用时只能省略尾部的实参:window = screen(,,'?')//错误;window =
screen('?');//调用screen('?',80,' ')
\item 在给定的作用域中,一个形参只能被赋予一次默认实参。函数的后续声明只能为
那些之前没有默认值的形参添加默认实参,而且该形参右侧的所有形参必须都有
默认实参
\item p213C++primer
\item 内联函数(inline)可以避免函数调用的开销,通常就是在每个调用点上内联的展
开;在返回类型前加上inline可将函数声明成内联函数
\item constexpr函数是指能用于常量表达式的函数,其定义需满足:1. 函数的返回类
型和所有形参类型都得是字面值类型;2. 函数体中必须有且只有一条return语句
(可以包含其他语句,但这些语句不能执行任何操作:空语句,类型别名,
using声明等)
\item constexpr函数被隐式的指定为内联函数,constexpr函数不一定返回常量表达式
\item 最好把inline函数和constexpr函数放在头文件中
\item c中的name.h文件在c++中为cname文件
\end{itemize}
\subsection{调试帮助}
\label{sec-4-4}
\begin{itemize}
\item 预处理宏实际上是一个预处理变量
\item assert是一种预处理宏:assert(expr),首先对expr求值,如果表达式为假(即
0),assert输出信息并终止程序的执行,否则什么也不做
\item assert宏定义在cassert中,预处理名字由预处理器而非编译器管理。因此我们可
以直接使用预处理名字而无需using声明
\item 和预处理变量一样,宏名字在程序中必须唯一
\item assert的行为依赖一个名为NDEBUG的预处理变量的状态,如果定义了NDEBUG,则
assert什么也不做。默认状态下没有定义NDEBUG
\item \uline{\_ func \_} 输出当前调试的函数的名字,由编译器定义
\item 预处理器定义的名字:\_\_ FILE \uline{\_ , \_} LINE \uline{\_ , \_} TIME \uline{\_ , \_} DATE \_\_
\end{itemize}
\subsection{函数匹配}
\label{sec-4-5}
\begin{itemize}
\item 函数匹配三步:1. 确定候选函数;2. 确定可行函数;3. 寻找最佳匹配
\item 候选函数:1. 与被调用的函数同名;2. 其声明在调用点可见
\item 可行函数:1. 其形参数量与本次调用所提供的实参数量相同;2. 每个实参类型
与对应的形参类型相同,或者能转换成形参的类型
\item 寻找最佳匹配,有且只有一个函数满足下列条件:1. 该函数对每个实参的匹配都
不劣于其它可行函数需要的匹配;2. 至少有一个实参的匹配由于其它可行函数提
供的匹配。如果无这样的函数,则该调用是错误的
\item 调用重载函数时,尽量避免强制类型转换
\item 小整型一般会提升成int类型或更大的整数类型
\item 所有算术类型转换的级别都一样
\item 编译器可以通过实参是否是常量来决定选择哪个函数(针对引用和指针)
\end{itemize}
\subsection{函数指针}
\label{sec-4-6}
\begin{itemize}
\item 函数指针指向的是函数而非对象。和其他指针一样,函数指针指向某种特定(函
数)类型。函数类型由他的返回类型和形参类型决定, \textbf{与函数名无关}
\item 声明一个指向函数的指针,只需用指针替换函数名即可:bool (*pf)(const
string \&);//未初始化的指针pf,*pf两端的括号必不可少(否则成了声明一个名
为pf的函数,该函数返回bool *)
\item 当我们把函数名作为一个值使用时,该函数自动转换成指针:pf =
lengthCompare; pf = \&lengthCompare;//等价;取地址符是可选的
\item 我们能直接使用指向函数的指针调用该函数,无需提前解引用指针:bool b1 =
pf("fuck"); bool b2 = (*pf)("fuck"); bool b3 =
lengthCompare("fuck");//三者等价
\item 指向不同函数类型的指针之间不存在转换规则。可以为函数指针赋一个nullptr或
者值为0的整型常量表达式
\item 重载函数的指针
\item 和数组类似,形参可以是指向函数的指针,此时,形参看起来是函数类型,实际
上是当作指针用(自动转换成指针)
\item decltype返回函数类型,不会将函数类型自动转成指针类型
\item 和数组类似,不能返回一个函数,但是可以返回一个指向函数类型的指针
\item 返回执行函数的指针:和函数类型的实参不同,返回类型不会自动的转换成指针,
我们必须显示的将返回类型指定为指针
\item 定义:int a = b =1;//不合法,此时b未定义;int b, a = b = 1;//可以
\end{itemize}

\section{类}
\label{sec-5}
\subsection{定义抽象数据类型}
\label{sec-5-1}
\begin{itemize}
\item 类的基本思想是数据抽象和封装。数据抽象是一种依赖于接口和实现分离的编程
技术;封装则实现了类的接口和实现的分离
\item 类设计者必须充分关注使用该类的程序员需要的功能
\item 类的成员函数的声明必须在类的内部,定义则既可以在类的内部也可以在类的外
部。定义在类内部的函数是隐式的inline函数
\item 任何对类成员的直接访问都被看作this的隐式引用。任何自定义名为this的参数
或变量(不在类中也不行)的行为都是非法的。我们可以在成员函数体内部使用
this
\item this 是一个常量指针,我们不允许改变this中保存的地址。当一个类对象使用成
员函数时,编译器把该对象的地址传给this
\item 默认情况下,this的类型是指向类类型非常量版本的常量指针
\item const成员函数:string isbn() const\{return bookNo;\}const的作用是修改隐式
this指针的类型,表示this是一个指向常量的指针,p231C++primer
\item 尽管this是隐式的,但是他仍然要遵循初始化规则,意味着(默认情况下)我们
不能把this对象绑定到一个常量对象上,这使得我们不能在一个常量对象上调用
普通的成员函数(因为一旦调用,则一般情况下需要访问类的数据成员,此时需要
利用this指针,而this指针需要初始化,但他是指向非常量的常量指针)
\item 常量对象,以及常量对象的引用或指针只能调用常量成员函数
\item 编译器处理类:1. 首先编译成员的声明; 2. 然后才轮到成员函数体。因此,成
员函数可以随意使用类中的其它成员而无需在意这些成员出现的次序
\item IO类属于不能被拷贝的类型,因此我们只能通过引用来传递他们
\item 默认情况下,拷贝类的对象其实拷贝的是对象的数据成员
\item 类通过一个或几个构造函数来控制其对象的初始化过程。无论何时只要类的对象
被创建,就会执行构造函数
\item 构造函数的名字和类名相同, \textbf{构造函数没有返回类型} 。构造函数可以重载
\item 构造函数不能被声明成const的,当我们创建一个类的const对象时,知道构造函
数完成初始化过程,对象才能真正获得其“常量”属性
\item 编译器创建的构造函数叫 \textbf{合成的默认构造函数} ,对大多数类来说,合成的默
认构造函数按照如下规则初始化类的数据成员:1. 如果存在类内的初始值,用它
来初始化成员;2. 否则,默认初始化该成员
\item 编译器只有在发现类中没有声明任何构造函数时,才会自动生成默认构造函数
\item 如果已经定义了其它构造函数,那么亦必须定义一个默认构造函数:
SalesData() = default;//这个函数的作用完全等同于之前使用的合成默认构造
函数
\end{itemize}

\subsection{访问控制与封装}
\label{sec-5-2}
\begin{itemize}
\item 使用class和struct定义类唯一的区别是默认的访问权限:class默认private;
struct默认public
\item 类允许其它类或者函数访问它的非公有成员,使用friend(友元)进行函数声明:
友元声明只能在类定义的内部,但是在类内出现的具体位置不限(不是类的成员,
不受访问控制的约束)
\item 封装的好处:1. 确保用户代码不会无意间破坏封装对象的状态;2. 类的具体实
现可以随时修改,而无需调整用户级别的代码(类的定义发生变化时,使用该类
的源文件需要重新编译)
\item 友元声明仅仅指定了访问的权限,而非通常意义上的函数声明
\item 定义一个类型成员:类可以定义某种类型在类中的别名,类型名字也存在访问权
限。用来定义类型的成员必须先定义后使用,这和普通成员不同!!!!!!
\item 使用mutable关键字声明可变数据成员,使其永远不是const,即使对象是const,
const函数也可以改变其值
\item 类内初始值必须使用=得初始化形式或者花括号括起来的直接初始化形式
\item 返回引用的函数是左值的
\item 一个const成员函数如果以引用的形式返回*this,则它的返回类型将是常量引用
(即使对象本身是非常量的)
\item 对于两个类来说,即使成员完全一样,只要类名不同就是不同的类
\item 我们可以只声明类但是暂时不定义它(前向声明),在其声明之后定义之前他是
一个不完全类型。只能在两种情况下使用:1. 可以定义指向这种类型的指针或引
用;2.亦可以声明(但是不能定义)以不完全类型作为参数或返回类型的函数
\item \textbf{类的成员类型不能是该类自己,但是类中可以包含指向他自身的引用或者指针}
\item 类还可以把其它类定义成友元,或者把已经定义过的类的成员函数定义成友元。
友元函数能定义在类的内部,这样的函数是隐式内联的
\item 友元关系不存在传递性
\item p252C++primer
\end{itemize}
\subsection{类的作用域}
\label{sec-5-3}
\begin{itemize}
\item 在类的作用域之外,普通的数据和函数成员只能由对象,引用或者指针使用成员
访问运算符来访问
\item 函数的返回类型通常出现在函数名之前,若函数定义在类的外部,则类中定义的
类型成员是函数的返回类型的话,必须要指定其所在的类 p253C++primer
\item 编译器处理完类中的全部声明后才回处理成员函数的定义
\item 不能在类中定义已经使用过(定义过)的类型别名
\item 使用::height在类中指明一个全局height,而非类中的height
\item 构造函数初始化类的数据成员和为数据成员赋值的区别p258C++primer
\item 如果成员是const或者是引用,或者当成员属于某种类类型且该类没有定义默认构
造函数时,必须将成员初始化
\item 成员的初始化顺序与他们在类中的定义顺序一致,构造函数初始值列表中的初始
值的位置关系不会影响实际的初始化顺序
\item \textbf{如果一个构造函数为所有参数都提供了默认实参,则他实际上也定义了默认构造
函数}
\item C++11中可以使用委托构造函数p261C++primer
\item 如果定义了其它构造函数,最好也提供一个默认构造函数
\item SalesData obj();//定义了一个函数而非对象
\item 如果构造函数直接受一个实参,则它实际定义了转换为此类类型的隐式转换机制,
称之为转换构造函数:可以从构造函数的参数类型向类类型隐式转换
\item \textbf{只允许一步类类型转换}
\item 构造函数前加上explicit可以阻止 \textbf{隐式} 的向类类型的转换:只对一个实参的构造
函数有效。只能在类内声明构造函数时使用explicit关键字, \textbf{在类外部定义时
不能重复}
\item explicit构造函数只能用于直接初始化,不能用于拷贝初始化(=形式)
\item 编译器不会将explicit构造函数用于隐式转换过程,但是我们可以使用这样的构
造函数显示的强制进行转换
\item 当在类的外部定义静态成员时,不能重复static关键字
\item 从类名开始,一条定义语句的剩余部分就都在类的作用域之内了
\item 类的静态成员和普通成员的区别:静态成员可以是不完全类型(在一个类内定义
此类类型的static对象成员;可以使用静态成员作为默认实参
\end{itemize}
\section{顺序容器}
\label{sec-6}
\begin{itemize}
\item string和vector将元素保存在连续的内存空间
\item list,forward\_ list 在容器中的任何位置添加和删除元素都很快,代价是,不
支持随机访问;额外内存开销很大
\item 通常,使用vector时最好的选择
\item 容器:顺序容器,关联容器,无序容器。容器皆定义为模板类
\item a.swap(b) == swap(a, b)//交换a,b的元素
\item forward\_ list迭代器不支持递减运算符(--)
\item 严格来说,string对象不属于容器类型,但是string支持很多与容器类型类似的
操作
\item 所有标准库容器都支持下标操作,但是其中只有少数几种才同时支持下标运算符
\item 迭代器范围是左闭合区间(begin, end)
\item 迭代器begin和end必须指向相同的容器,end可以与begin指向相同的容器,但是
end不能指向begin之前的位置
\item 对一个反向迭代器使用++操作会得到上一个元素
\item 当将一个容器初始化为另一个容器的拷贝时,两个容器的容器类型和元素类型必
须都相同;当传递迭代器参数来拷贝一个范围时,容器类型可以不同,容器中的
元素类型也可以不同,只要能转换成目标容器中的元素类型就行
\item 只有顺序容器(除了array)的构造函数才接受大小参数,关联容器并不支持
(string不支持 string str(n)的形式)
\item 与内置数组一样,标准库array的大小也是类型的一部分:array<int, 42> arr;
\item 数组不支持拷贝和赋值,但是array支持(类型和大小都相同)
\item array类型不支持assign,不允许用花括号包围的值列表进行赋值(可以用之来初
始化)
\item swap只是交换了两个容器的内部数据结构,不对任何元素进行拷贝,删除或插入
操作,可在常数时间内完成;对string调用swap会导致迭代器,引用和指针失效
(对其余容器不会);swap两个数组会真正交换他们的内容,时间O(n)
\item 关系运算符:除了无序关联容器外的所有容器都支持关系运算符(>, >=, < ,
<=),两个运算对象必须具有相同的类型,和相同的元素类型
\item 顺序容器和关联容器的不同之处在于两者组织元素的方式,这些不同直接关系到
了元素如何存储,访问,添加和删除
\item forward\_ list不支持push\_ back和emplace\_ back;vector和string不支持
push\_ front和emplace\_ front
\item \textbf{容器元素是拷贝,当我们将一个对象插入到容器中时,实际上放入容器中的是对
象值的一个拷贝,而不是对象本身}
\item emplace操作:直接构造而不是拷贝元素,传递给emplace函数的参数必须与元素
类型的构造函数相匹配
\item 访问容器元素:包括array在内的每个顺序容器都有一个front函数,而除
forward\_ list之外的所有顺序容器都有一个back成员函数,这两个操作分别返回
首元素和尾元素的引用;若容器为空,函数行为未定义
\item 不能递减forward\_ list迭代器
\item 访问成员函数返回的是引用:front,back,下标和at函数。使用下标不负责检查
是否越界,at函数检查
\item pop\_ back和pop\_ front函数返回void
\item 如果容器是vector或string,且存储空间被重新分配,则指向容器的迭代器,引
用和指针都会失效
\item 当我们添加或删除vector或string的元素后,或在deque的首元素之外的任何位置
添加删除元素后,原来end返回的迭代器都会失效:添加或删除元素的循环程序必
须反复调用end,而不能在循环之前保存end返回的迭代器,一直当迭代器末尾使
用
\item 三个顺序容器的适配器:stack,queue,priority\_ queue
\item 适配器是标准库的一个通用概念,容器,迭代器和函数都有适配器。一个适配器
是一种机制,能使某种事物的行为看起来像另外一种事物一样。一个容器适配器
接受一种已有的容器类型,使其行为看起来像一种不同的类型
\item 每个适配器有两个构造函数:1. 默认构造函数默认构造函数创建一个空对象;
\begin{enumerate}
\item 接受一个容器的构造函数拷贝该容器来初始化适配器
\end{enumerate}
\item 默认情况下,stack和queue基于deque实现,priority\_ queue基于vector实现
\end{itemize}
\section{泛型算法}
\label{sec-7}
\begin{itemize}
\item 实现了一些经典算法的接口,泛型指的是他们可以用于不同类型的元素和多种容
器类型
\item 一般情况下,这些算法不直接操作容器,而是遍历由两个迭代器指定的一个元素
范围
\item \textbf{迭代器令算法不依赖于容器} ,但算法依赖于元素类型的操作:泛型算法本身不
会执行容器的操作,他们只会运行与迭代器之上,执行迭代器的操作。必要的编
程假定:算法永远不会改变底层容器的大小
\item 标准库在iterator中还定义了额外几种迭代器:插入迭代器,流迭代器,反向迭
代器,移动迭代器
\item 迭代器类别:输入迭代器,输出迭代器,前向迭代器,双向迭代器,随机访问迭
代器
\end{itemize}
\section{关联容器}
\label{sec-8}
\begin{itemize}
\item 关联容器和顺序容器的区别:关联容器中的元素是按关键字来保存和访问的;顺
序容器中的元素是按他们在容器中的位置来顺序保存和访问的
\item map是关键字-值对的集合,通常被称为关联数组,存储的元素为pair类型的对象,
用.first和.second获取键和值
\item 关联容器支持普通的容器操作但是不支持顺序容器的位置相关的操作;关联容器
的迭代器都是双向的
\item map和set的关键字唯一,multiset和multimap可重复
\item 关联容器额外的类型别名:key\_ type,mapped\_ type,value\_ type
\item 关键字部分是const的
\item *一个map的value\_ type是一个pair,我么可以改变pair的值,但是不能改变关键
字成员的值
\item set的迭代器是const的
\item 插入元素insert(), 接受一对迭代器或一个初始化列表,返回一个pair(第一个
成员是一个迭代器,指向具有给定关键字的元素,second元素是一个bool值,指
出元素插入成功还是已经在容器中)
\item 向multiset和multimap添加元素,使用insert返回一个指向新元素的迭代器(无
需返回bool值)
\item map和unordered\_ map可以使用下标运算符和at函数,set类型不支持下标,
multi不支持下标。使用下标运算符,如果关键字不再map中,会为他创建一个元
素并插入到map中(at函数不会,只会返回异常),关联值将进行值初始化(int
为0),只能为非const的map使用下标操作
\item 当对一个map进行下标操作时,会得到一个mapped\_ type对象,但是当解引用一个
map迭代器时,会得到一个value\_ type。map的下标运算符返回一个左值
\item 如果multiset和multimap具有多个元素具有给定关键字,则这些元素在 \textbf{容器}
  中会相邻存储
\item 无序关联容器:这些容器不是使用比较运算符来组织元素,而是使用一个哈希函
数和关键字类型的==运算符
\item 无序容器存储上组织为一组桶,无序容器使用一个哈希函数把元素映射到一个桶。
无序容器的性能依赖于哈希函数的质量和桶的数量和大小
\item 无序容器对关键字类型的要求:内置类型(指针类型),string和只能指针,可
以直接定义
\item 我们不能 \textbf{直接} 定义关键字类型为自定义类类型的无序容器,需要重载==运算
符
\end{itemize}

\section{动态内存}
\label{sec-9}
\begin{itemize}
\item 动态分配的:在自由空间中分配的对象(new),知道被显示释放(delete)或程
序结束才回销毁
\item 看默认初始化和值初始化:对于内置类型,值初始化有着良好定义的值,而默认
初始化的值则是未定义的
\item new:从自由空间中分配内存,new T分配并构造类型为T的对象,返回指向该对象
的指针。如果T是数组类型,new返回指向数组首元素的指针。new [n] T分配n个
类型为T的对象,并返回指向数组首元素的指针。默认情况下,分配的对象进行默
认初始化
\item delete:释放new分配的内存。delete p释放对象,delete []p释放p指向的数组。
p可以为空,或者指向new分配的内存
\item 自由空间:程序可用的内存池,保持动态分配的对象
\item p40,p88 C++primer
\end{itemize}
% Emacs 24.3.1 (Org mode 8.2.10)
\end{document}